


\chapter{Introduction}

    \section{Introduce the topic of deep neural networks and their applications in image segmentation/localization}

        Convolutional Neural Networks (CNN) sind zu einem grundlegenden Werkzeug in der
        Computer Vision und Bildverarbeitung geworden. Sie sind eine spezielle Art von tiefen neuronalen Netzen,
        die entwickelt wurden, um hierarchische und abstrakte Merkmale aus Bildern für Klassifizierungs- und
        Lokalisierungsaufgaben zu extrahieren. Die Kernidee hinter CNNs ist die Verwendung von Faltungsschichten
        mit lernfähigen Filtern um lokale Merkmale aus verschiedenen Regionen des Eingabebildes zu extrahieren.
        Diese Filter gleiten über das Eingabebild und berechnen Punktprodukte zwischen den Filterwerten und den
        entsprechenden Pixelwerten.

        CNNs haben bemerkenswerte Erfolge in einem breiten Spektrum von Anwendungen erzielt, darunter
        Bildklassifikation, Objekterkennung und Segmentierung. Bei der Bildsegmentierung wird ein Bild in mehrere
        Segmente oder Regionen unterteilt, die jeweils einem bestimmten Objekt oder Hintergrund im Bild entsprechen.
        CNNs eignen sich besonders gut für Bildsegmentierungsaufgaben, da sie lernen können, komplexe Merkmale und
        Muster aus Bildern zu extrahieren und Objektgrenzen und räumliche Beziehungen zwischen Objekten zu erkennen.

        In den letzten Jahren haben CNN-basierte Verfahren ihre Leistungsfähigkeit in verschiedenen
        Bildsegmentierungsanwendungen unter Beweis gestellt, darunter medizinische Bildgebung, Satellitenbilder und
        autonomes Fahren. In der medizinischen Bildgebung wurden CNNs beispielsweise zur Erkennung und Segmentierung
        von Tumoren, Hirnverletzungen und Anomalien in verschiedenen Organen eingesetzt. Im Bereich des autonomen
        Fahrens wurden CNN-basierte Modelle zur Erkennung von Fahrspuren und Hindernissen sowie zur semantischen
        Segmentierung der Umgebung eingesetzt.

    \section{Outline the purpose of your paper and the approach you will use to train your model}

        Dieser Beitrag gibt einen Überblick über die jüngsten Fortschritte bei CNN-basierten Methoden zur
        Bildsegmentierung und -lokalisierung. Wir werden die wichtigsten Ideen und Techniken hinter CNNs diskutieren,
        einschließlich der Verwendung von Faltungsschichten, Pooling-Schichten und vollständig verbundenen Schichten.
        Wir werden auch die Herausforderungen und Grenzen von CNNs aufzeigen, wie z.B. die Notwendigkeit großer
        Mengen annotierter Trainingsdaten und erheblicher Hardware-Ressourcen für Training und Einsatz.

        Um unser CNN-basiertes Modell zur Bildsegmentierung zu trainieren, werden wir einen großen annotierten
        Datensatz und ein Deep Learning Framework wie TensorFlow oder PyTorch verwenden. Wir werden die Eingabebilder
        vorverarbeiten, indem wir ihre Pixelwerte normalisieren und Datenanreicherungstechniken wie Rotation,
        Translation und Skalierung anwenden. Wir trainieren unser Modell mit einer Verlustfunktion, die den
        Unterschied zwischen der vorhergesagten und der tatsächlichen Segmentierung misst, wie z.B.
        Cross-Entropie-Verlust oder Dice-Verlust. Wir werden Backpropagation und Stochastic Gradient Descent (SGD)
        oder Varianten davon verwenden, um die Parameter unseres Modells zu aktualisieren und den Verlust zu minimieren.

        Big bullshit incoming:

        Um die Leistung unseres Modells zu bewerten, werden wir Standardmetriken wie Intersection over Union (IoU),
        Dice-Koeffizient und Pixelgenauigkeit verwenden. Außerdem werden wir die Leistung unseres Modells mit
        modernen Methoden vergleichen und seine Stärken und Schwächen analysieren. Abschließend werden wir zukünftige
        Forschungsrichtungen und potentielle Anwendungen von CNN-basierten Methoden in der Bildsegmentierung und
        -lokalisierung diskutieren.
