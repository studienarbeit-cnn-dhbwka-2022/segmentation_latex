\chapter{Deep neural networks}

    \section{Explain the concepts of artificial neural networks and deep learning}

        Künstliche neuronale Netze (ANN) sind algorithmische Modelle, die das biologische Verhalten von Neuronen
        im menschlichen Gehirn nachahmen. ANNs bestehen aus vernetzten Knoten, die Informationen in mehreren
        Schichten verarbeiten. In einem einfachen neuronalen Feed-Forward-Netzwerk fließt die Information
        unidirektional von der Eingabeschicht zur Ausgabeschicht, wobei jede Schicht eine bestimmte Funktion
        ausführt. ANNs können trainiert werden, um Muster und Beziehungen in den Daten durch einen Prozess zu 
        lernen, der als überwachtes Lernen bezeichnet wird, bei dem das Netzwerk mit markierten Trainingsbeispielen
        konfrontiert wird und seine internen Parameter anpasst, um den Fehler zwischen seiner Ausgabe und den
        korrekten Markierungen zu minimieren.

        Deep Learning ist ein Teilgebiet des maschinellen Lernens, das ANNs mit mehreren verborgenen Schichten
        umfasst. Tiefe neuronale Netze (DNN) können hierarchische Darstellungen von Daten lernen, wobei jede
        Schicht abstraktere Merkmale lernt als die vorhergehende Schicht. DNNs haben sich bei verschiedenen
        Aufgaben als besonders leistungsfähig erwiesen, z. B. in den Bereichen maschinelles Sehen, Verarbeitung
        natürlicher Sprache und Spracherkennung. Insbesondere DNNs mit Faltungsschichten, so genannte Convolutional
        Neural Networks (CNNs), sind sehr effektiv für bildbezogene Aufgaben wie Bildklassifikation, Objekterkennung
        und Bildsegmentierung.
        
        Der Erfolg des Deep Learning lässt sich auf mehrere Faktoren zurückführen, darunter die Verfügbarkeit großer
        Mengen an markierten Daten, Fortschritte bei der Rechenleistung und der Parallelverarbeitung sowie
        Verbesserungen bei Optimierungsalgorithmen wie dem stochastischen Gradientenabstieg. Deep Learning hat auch
        zur Entwicklung neuer Techniken geführt, z. B. Transfer Learning, bei dem zuvor trainierte Modelle für
        bestimmte Aufgaben verfeinert werden, und Generative Adversarial Networks (GANs), die realistische Bilder
        und Videos erzeugen können.

        Zusammenfassend lässt sich sagen, dass die Konzepte der ANN und des Deep Learning den Bereich der künstlichen
        Intelligenz revolutioniert haben, indem sie Maschinen in die Lage versetzen, komplexe Aufgaben zu bewältigen,
        von denen man früher glaubte, dass sie ausschließlich dem Menschen vorbehalten seien.

    \section{Discuss the architecture of convolutional neural networks (CNNs) and their role in image segmentation/localization}

        Convolutional Neural Networks (CNNs) sind eine Art Deep Neural Networks, die den Bereich des maschinellen
        Sehens revolutioniert haben. CNNs sind darauf ausgelegt, visuelle Eingaben wie Bilder oder Videos zu
        analysieren und Merkmale zu extrahieren, die für eine bestimmte Aufgabe relevant sind. CNNs sind besonders
        effektiv bei bildbezogenen Aufgaben wie Bildklassifikation, Objekterkennung und Bildsegmentierung.

        Die Architektur eines CNN besteht aus einer Folge von Schichten, die das Eingangsbild in ein vorhergesagtes
        Ausgangsbild umwandeln. Die erste Schicht eines CNN ist normalerweise eine Faltungsschicht, die eine Reihe
        von Filtern auf das Eingangsbild anwendet. Jeder Filter ist eine kleine Matrix von Gewichten, die mit dem
        Eingabebild gefaltet werden, um lokale Merkmale zu extrahieren. Faltungsschichten können lernen,
        grundlegende Muster wie Kanten, Ecken oder Flecken zu erkennen, die häufig in Bildern vorkommen.
        
        Die Ausgabe einer Faltungsschicht wird dann durch eine Aktivierungsfunktion wie ReLU oder Sigmoid geleitet,
        die dem Modell Nichtlinearität verleiht. Die Aufgabe der Aktivierungsfunktion besteht darin, die Ausgabe
        der Faltungsschicht in eine komplexere und aussagekräftigere Merkmalsdarstellung umzuwandeln. Diese
        Merkmalsdarstellung wird dann durch eine Pooling-Schicht geleitet, die die räumlichen Dimensionen der
        Merkmalskarte reduziert, während die wesentlichen Informationen erhalten bleiben. Pooling-Schichten können
        auch eine Form von Translationsinvarianz einführen, was bedeutet, dass das Modell dasselbe Objekt unabhängig
        von seiner Position im Bild erkennen kann.

        Die Ausgabe der Pooling-Schicht wird dann in einen weiteren Satz von Faltungsschichten, Aktivierungsschichten
        und Pooling-Schichten eingespeist. Die Anzahl dieser Schichten kann je nach Komplexität der Aufgabe und Größe
        des Datensatzes variieren. Schließlich wird die Ausgabe der letzten Pooling-Schicht abgeflacht und durch eine
        oder mehrere vollständig verbundene Schichten geleitet, die die endgültige Klassifizierung oder Segmentierung
        vornehmen.

        Bei der Bildsegmentierung können CNNs lernen, jedes Pixel eines Bildes einer bestimmten Klasse oder Kategorie
        zuzuordnen. In der medizinischen Bildgebung können CNNs beispielsweise zur Unterscheidung von Tumoren oder
        Läsionen von gesundem Gewebe eingesetzt werden. Die Fähigkeit von CNNs, räumliche Merkmale aus Bildern zu
        lernen, macht sie besonders geeignet für Bildsegmentierungsaufgaben. Durch die Nutzung der räumlichen Kohärenz
        des Bildes können CNNs hochpräzise Segmentierungen durchführen, die herkömmliche Segmentierungsmethoden
        übertreffen.
    
    

    \section{Describe the different layers of a CNN and how they are used to extract features from images}

        Faltungsschichten sind das Herzstück von CNNs. Diese Schichten lernen, relevante Merkmale aus dem Eingabebild
        zu extrahieren, indem sie eine Reihe von adaptiven Filtern auf das Bild anwenden. Jeder Filter gleitet über
        das gesamte Eingabebild und führt eine elementweise Multiplikation mit anschließender Summation durch. Das
        Ergebnis dieser Operation ist ein einzelner Skalarwert, der den Grad der Ähnlichkeit zwischen dem Filter und
        der entsprechenden Region des Eingabebildes darstellt. Diese Skalarwerte werden dann zu einer Merkmalskarte
        angeordnet.

        Nach jeder Faltungsschicht werden Aktivierungsschichten hinzugefügt, um dem Modell Nichtlinearität zu
        verleihen. Ohne diese Schichten wäre das Modell auf lineare Transformationen beschränkt, die nicht
        leistungsfähig genug sind, um komplexe nichtlineare Beziehungen in den Daten zu modellieren.
        Aktivierungsfunktionen wie ReLU, Sigmoid und tanh werden üblicherweise in CNNs verwendet.
        
        Pooling-Layer werden verwendet, um die räumlichen Dimensionen der Merkmalskarten zu reduzieren, während die
        wesentlichen Informationen erhalten bleiben. Diese Schichten unterteilen die Merkmalskarten in nicht
        überlappende Regionen und wenden auf jede Region eine Reduktionsoperation an, wie z. B. Max- oder
        Average-Pooling. Dadurch wird die Anzahl der Modellparameter reduziert und das Modell recheneffizienter.

        Voll verknüpfte Schichten werden normalerweise am Ende des CNN hinzugefügt, um das Bild zu klassifizieren oder
        die Segmentierungsmaske vorherzusagen. Diese Schichten nehmen die abgeflachten Merkmalskarten aus den
        vorhergehenden Schichten und lassen sie durch einen Satz vollständig verbundener Neuronen laufen, um die
        endgültige Vorhersage zu treffen. Im Fall der Bildklassifikation stellt die Ausgabe der vollständig
        verbundenen Schicht die Wahrscheinlichkeitsverteilung über die verschiedenen Klassen dar. Im Falle der
        Bildsegmentierung stellt die Ausgabe der vollständig verbundenen Schicht die Wahrscheinlichkeit dar, dass
        jedes Pixel zu dem interessierenden Objekt gehört.

        Insgesamt ermöglicht die Kombination dieser verschiedenen Schichten den CNNs, komplexe hierarchische
        Repräsentationen des Eingabebildes zu erlernen, wodurch sie für Aufgaben wie Bildklassifikation,
        Objekterkennung und Segmentierung gut geeignet sind.
    